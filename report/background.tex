\section{Background}
\label{section:background}

Wireguard~\cite{wireguard} is a novel security protocol that allows for building secure 
Virtual Private Networks on top of layer 3. Wireguard uses the notion of a cryptorouting 
in which the forwarding of a packet occurs based on the public key of a peer. Wireguard 
is a 1-RTT protocol, meaning that it requires (in the best case) a single round-trip to 
negotiate the keys for securing the data plane traffic. In addition, Wireguard introduces 
the concept of cryptographic cookies which can be used during the heavy load time instants 
to protect the server from denial-of-service attacks. Finally, Wireguard uses the latest 
cryptographic algorithms such as Authenticated Encryption with Authentication Data (AEAD) 
ChaCha20 algorithm and compact 256-bit elliptic curve X25519. In what follows we will 
describe the setup of Wireguard between two nodes deployed on virtual machines.